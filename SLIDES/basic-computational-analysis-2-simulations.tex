\documentclass[aspectratio=169]{beamer}\usepackage[]{graphicx}\usepackage[]{xcolor}
% maxwidth is the original width if it is less than linewidth
% otherwise use linewidth (to make sure the graphics do not exceed the margin)
\makeatletter
\def\maxwidth{ %
  \ifdim\Gin@nat@width>\linewidth
    \linewidth
  \else
    \Gin@nat@width
  \fi
}
\makeatother

\definecolor{fgcolor}{rgb}{0.345, 0.345, 0.345}
\newcommand{\hlnum}[1]{\textcolor[rgb]{0.686,0.059,0.569}{#1}}%
\newcommand{\hlsng}[1]{\textcolor[rgb]{0.192,0.494,0.8}{#1}}%
\newcommand{\hlcom}[1]{\textcolor[rgb]{0.678,0.584,0.686}{\textit{#1}}}%
\newcommand{\hlopt}[1]{\textcolor[rgb]{0,0,0}{#1}}%
\newcommand{\hldef}[1]{\textcolor[rgb]{0.345,0.345,0.345}{#1}}%
\newcommand{\hlkwa}[1]{\textcolor[rgb]{0.161,0.373,0.58}{\textbf{#1}}}%
\newcommand{\hlkwb}[1]{\textcolor[rgb]{0.69,0.353,0.396}{#1}}%
\newcommand{\hlkwc}[1]{\textcolor[rgb]{0.333,0.667,0.333}{#1}}%
\newcommand{\hlkwd}[1]{\textcolor[rgb]{0.737,0.353,0.396}{\textbf{#1}}}%
\let\hlipl\hlkwb

\usepackage{framed}
\makeatletter
\newenvironment{kframe}{%
 \def\at@end@of@kframe{}%
 \ifinner\ifhmode%
  \def\at@end@of@kframe{\end{minipage}}%
  \begin{minipage}{\columnwidth}%
 \fi\fi%
 \def\FrameCommand##1{\hskip\@totalleftmargin \hskip-\fboxsep
 \colorbox{shadecolor}{##1}\hskip-\fboxsep
     % There is no \\@totalrightmargin, so:
     \hskip-\linewidth \hskip-\@totalleftmargin \hskip\columnwidth}%
 \MakeFramed {\advance\hsize-\width
   \@totalleftmargin\z@ \linewidth\hsize
   \@setminipage}}%
 {\par\unskip\endMakeFramed%
 \at@end@of@kframe}
\makeatother

\definecolor{shadecolor}{rgb}{.97, .97, .97}
\definecolor{messagecolor}{rgb}{0, 0, 0}
\definecolor{warningcolor}{rgb}{1, 0, 1}
\definecolor{errorcolor}{rgb}{1, 0, 0}
\newenvironment{knitrout}{}{} % an empty environment to be redefined in TeX

\usepackage{alltt}




\usetheme{default}
% Slide setup, colour independent

\usepackage{amsmath,amssymb,amsthm}
\usepackage[utf8]{inputenc}
\usepackage{colortbl}
\usepackage{bm}
\usepackage{xcolor}
\usepackage{dsfont}
\usepackage{setspace}
%\usepackage{subfigure}
% To use \ding{234} and the like
\usepackage{pifont}
% To cross reference between slide files
\usepackage{zref-xr,zref-user}
% Use something like
% \zexternaldocument{fileI}
% in the tex files. And cite using \zref instead of \ref
\usepackage{booktabs}

%\usepackage{transparent}

% Fields and the like
\def\IC{\mathbb{C}}
\def\IF{\mathbb{F}}
\def\II{\mathbb{I}}
\def\IJ{\mathbb{J}}
\def\IM{\mathbb{M}}
\def\IN{\mathbb{N}}
\def\IP{\mathbb{P}}
\def\IR{\mathbb{R}}
\def\IZ{\mathbb{Z}}
\def\11{\mathds{1}}


% Bold lowercase
\def\ba{\bm{a}}
\def\bb{\bm{b}}
\def\bc{\bm{c}}
\def\bd{\bm{d}}
\def\be{\bm{e}}
\def\bf{\bm{f}}
\def\bh{\bm{h}}
\def\bi{\bm{i}}
\def\bj{\bm{j}}
\def\bk{\bm{k}}
\def\bn{\bm{n}}
\def\bp{\bm{p}}
\def\br{\bm{r}}
\def\bs{\bm{s}}
\def\bu{\bm{u}}
\def\bv{\bm{v}}
\def\bw{\bm{w}}
\def\bx{\bm{x}}
\def\by{\bm{y}}
\def\bz{\bm{z}}

% Bold capitals
\def\bB{\bm{B}}
\def\bD{\bm{D}}
\def\bE{\bm{E}}
\def\bF{\bm{F}}
\def\bG{\bm{G}}
\def\bI{\bm{I}}
\def\bL{\bm{L}}
\def\bN{\bm{N}}
\def\bP{\bm{P}}
\def\bR{\bm{R}}
\def\bS{\bm{S}}
\def\bT{\bm{T}}
\def\bX{\bm{X}}

% Bold numbers
\def\b0{\bm{0}}

% Bold greek
\bmdefine{\bmu}{\bm{\mu}}
\def\bphi{\bm{\phi}}
\def\bvarphi{\bm{\varphi}}
\def\bPi{\bm{\Pi}}
\def\bGamma{\bm{\Gamma}}

% Bold red sentence
\def\boldred#1{{\color{red}\textbf{#1}}}
\def\defword#1{{\color{orange}\textbf{#1}}}

% Caligraphic letters
\def\A{\mathcal{A}}
\def\B{\mathcal{B}}
\def\C{\mathcal{C}}
\def\D{\mathcal{D}}
\def\E{\mathcal{E}}
\def\F{\mathcal{F}}
\def\G{\mathcal{G}}
\def\H{\mathcal{H}}
\def\I{\mathcal{I}}
\def\L{\mathcal{L}}
\def\M{\mathcal{M}}
\def\N{\mathcal{N}}
\def\P{\mathcal{P}}
\def\R{\mathcal{R}}
\def\S{\mathcal{S}}
\def\T{\mathcal{T}}
\def\U{\mathcal{U}}
\def\V{\mathcal{V}}

% Adding space for prime (') where needed
\def\pprime{\,'}
% Adding space for star (\star) where needed
\def\pstar{{\,\star}}

% tt font for code
\def\code#1{{\tt #1}}

% i.e., e.g.
\def\eg{\emph{e.g.}}
\def\ie{\emph{i.e.}}


% Operators and special symbols
\def\nbOne{{\mathchoice {\rm 1\mskip-4mu l} {\rm 1\mskip-4mu l}
{\rm 1\mskip-4.5mu l} {\rm 1\mskip-5mu l}}}
\def\cov{\ensuremath{\mathsf{cov}}}
\def\Var{\ensuremath{\mathsf{Var}\ }}
\def\Im{\textrm{Im}\;}
\def\Re{\textrm{Re}\;}
\def\det{\ensuremath{\mathsf{det}}}
\def\diag{\ensuremath{\mathsf{diag}}}
\def\nullspace{\ensuremath{\mathsf{null}}}
\def\nullity{\ensuremath{\mathsf{nullity}}}
\def\rank{\ensuremath{\mathsf{rank}}}
\def\range{\ensuremath{\mathsf{range}}}
\def\sgn{\ensuremath{\mathsf{sgn}}}
\def\Span{\ensuremath{\mathsf{span}}}
\def\tr{\ensuremath{\mathsf{tr}}}
\def\imply{$\Rightarrow$}
\def\restrictTo#1#2{\left.#1\right|_{#2}}
\newcommand{\parallelsum}{\mathbin{\!/\mkern-5mu/\!}}
\def\dsum{\mathop{\displaystyle \sum }}%
\def\dind#1#2{_{\substack{#1\\ #2}}}

\DeclareMathOperator{\GL}{GL}
\DeclareMathOperator{\Rel}{Re}
\def\Nt#1{\left|\!\left|\!\left|#1\right|\!\right|\!\right|}
\newcommand{\tripbar}{|\! |\! |}



% The beamer bullet (in base colour)
\def\bbullet{\leavevmode\usebeamertemplate{itemize item}\ }

% Theorems and the like
\newtheorem{proposition}[theorem]{Proposition}
\newtheorem{property}[theorem]{Property}
\newtheorem{importantproperty}[theorem]{Property}
\newtheorem{importanttheorem}[theorem]{Theorem}
%\newtheorem{lemma}[theorem]{Lemma}
%\newtheorem{corollary}[theorem]{Corollary}
\newtheorem{remark}[theorem]{Remark}
\setbeamertemplate{theorems}[numbered]
%\setbeamertemplate{theorems}[ams style]

%
%\usecolortheme{orchid}
%\usecolortheme{orchid}

\def\red{\color[rgb]{1,0,0}}
\def\blue{\color[rgb]{0,0,1}}
\def\green{\color[rgb]{0,1,0}}


% Get rid of navigation stuff
\setbeamertemplate{navigation symbols}{}

% Set footline/header line
\setbeamertemplate{footline}
{%
\quad p. \insertpagenumber \quad--\quad \insertsection\vskip2pt
}
% \setbeamertemplate{headline}
% {%
% \quad\insertsection\hfill p. \insertpagenumber\quad\mbox{}\vskip2pt
% }


\makeatletter
\newlength\beamerleftmargin
\setlength\beamerleftmargin{\Gm@lmargin}
\makeatother

% Colours for special pages
\def\extraContent{yellow!20}


%%%%%%%%%%%%%%%%%
\usepackage{tikz}
\usetikzlibrary{shapes,arrows}
\usetikzlibrary{positioning}
\usetikzlibrary{shapes.symbols,shapes.callouts,patterns}
\usetikzlibrary{calc,fit}
\usetikzlibrary{backgrounds}
\usetikzlibrary{decorations.pathmorphing,fit,petri}
\usetikzlibrary{automata}
\usetikzlibrary{fadings}
\usetikzlibrary{patterns,hobby}
\usetikzlibrary{backgrounds,fit,petri}


\usepackage{pgfplots}
\pgfplotsset{compat=1.6}
\pgfplotsset{ticks=none}

\usetikzlibrary{decorations.markings}
\usetikzlibrary{arrows.meta}
\tikzset{>=stealth}

% For tikz
\tikzstyle{cloud} = [draw, ellipse,fill=red!20, node distance=0.87cm,
minimum height=2em]
\tikzstyle{line} = [draw, -latex']


%%% For max frame images
\newenvironment{changemargin}[2]{%
\begin{list}{}{%
\setlength{\topsep}{0pt}%
\setlength{\leftmargin}{#1}%
\setlength{\rightmargin}{#2}%
\setlength{\listparindent}{\parindent}%
\setlength{\itemindent}{\parindent}%
\setlength{\parsep}{\parskip}%
}%
\item[]}{\end{list}}


% Make one image take up the entire slide content area in beamer,.:
% centered/centred full-screen image, with title:
% This uses the whole screen except for the 1cm border around it
% all. 128x96mm
\newcommand{\titledFrameImage}[2]{
\begin{frame}{#1}
%\begin{changemargin}{-1cm}{-1cm}
\begin{center}
\includegraphics[width=108mm,height=\textheight,keepaspectratio]{#2}
\end{center}
%\end{changemargin}
\end{frame}
}

% Make one image take up the entire slide content area in beamer.:
% centered/centred full-screen image, no title:
% This uses the whole screen except for the 1cm border around it
% all. 128x96mm
\newcommand{\plainFrameImage}[1]{
\begin{frame}[plain]
%\begin{changemargin}{-1cm}{-1cm}
\begin{center}
\includegraphics[width=108mm,height=76mm,keepaspectratio]{#1}
\end{center}
%\end{changemargin}
\end{frame}
}

% Make one image take up the entire slide area, including borders, in beamer.:
% centered/centred full-screen image, no title:
% This uses the entire whole screen
\newcommand{\maxFrameImage}[1]{
\begin{frame}[plain]
\begin{changemargin}{-1cm}{-1cm}
\begin{center}
\includegraphics[width=\paperwidth,height=\paperheight,keepaspectratio]
{#1}
\end{center}
\end{changemargin}
\end{frame}
}

% This uses the entire whole screen (to include in frame)
\newcommand{\maxFrameImageNoFrame}[1]{
\begin{changemargin}{-1cm}{-1cm}
\begin{center}
\includegraphics[width=\paperwidth,height=0.99\paperheight,keepaspectratio]
{#1}
\end{center}
\end{changemargin}
}

% Make one image take up the entire slide area, including borders, in beamer.:
% centered/centred full-screen image, no title:
% This uses the entire whole screen
\newcommand{\maxFrameImageColor}[2]{
\begin{frame}[plain]
\setbeamercolor{normal text}{bg=#2!20}
\begin{changemargin}{-1cm}{-1cm}
\begin{center}
\includegraphics[width=\paperwidth,height=\paperheight,keepaspectratio]
{#1}
\end{center}
\end{changemargin}
\end{frame}
}


\usepackage{tikz}
\usetikzlibrary{patterns,hobby}
\usepackage{pgfplots}
\pgfplotsset{compat=1.6}
\pgfplotsset{ticks=none}

\usetikzlibrary{backgrounds}
\usetikzlibrary{decorations.markings}
\usetikzlibrary{arrows.meta}
\tikzset{>=stealth}

\tikzset{
  clockwise arrows/.style={
    postaction={
      decorate,
      decoration={
        markings,
        mark=between positions 0.1 and 0.9 step 40pt with {\arrow{>}},
   }}}}


   %%%%%%%%%%%
% To have links to parts in the outline
\makeatletter
\AtBeginPart{%
  \addtocontents{toc}{\protect\beamer@partintoc{\the\c@part}{\beamer@partnameshort}{\the\c@page}}%
}
%% number, shortname, page.
\providecommand\beamer@partintoc[3]{%
  \ifnum\c@tocdepth=-1\relax
    % requesting onlyparts.
    \makebox[6em]{Part #1:} \textcolor{green!30!blue}{\hyperlink{#2}{#2}}
    \par
  \fi
}
\define@key{beamertoc}{onlyparts}[]{%
  \c@tocdepth=-1\relax
}
\makeatother%

\newcommand{\nameofthepart}{}
\newcommand{\nupart}[1]%
    {   \part{#1}%
        \renewcommand{\nameofthepart}{#1}%
        {
          \setbeamercolor{background canvas}{bg=orange!50}
          \begin{frame}{#1}%\partpage 
          \hypertarget{\nameofthepart}{}\tableofcontents%
          \end{frame}
        }
    }



\usecolortheme{orchid}
%% Listings
\usepackage{listings}
\definecolor{mygreen}{rgb}{0,0.6,0}
\definecolor{mygray}{rgb}{0.5,0.5,0.5}
\definecolor{mymauve}{rgb}{0.58,0,0.82}
\definecolor{mygold}{rgb}{1,0.843,0}
\definecolor{myblue}{rgb}{0.537,0.812,0.941}

\definecolor{mygold2}{RGB}{120,105,22}
\definecolor{mygrey2}{RGB}{50,50,50}

\definecolor{lgreen}{rgb}{0.6,0.9,.6}
\definecolor{lred}{rgb}{1,0.5,.5}

\lstloadlanguages{R}
\lstset{ %
  language=R,
  backgroundcolor=\color{black!05},   % choose the background color
  basicstyle=\footnotesize\ttfamily,        % size of fonts used for the code
  breaklines=true,                 % automatic line breaking only at whitespace
  captionpos=b,                    % sets the caption-position to bottom
  commentstyle=\color{mygreen},    % comment style
  escapeinside={\%*}{*)},          % if you want to add LaTeX within your code
  keywordstyle=\color{red},       % keyword style
  stringstyle=\color{mygold},     % string literal style
  keepspaces=true,
  columns=fullflexible,
  tabsize=4,
}
% Could also do (in lstset)
% basicstyle==\fontfamily{pcr}\footnotesize
\lstdefinelanguage{Renhanced}%
  {keywords={abbreviate,abline,abs,acos,acosh,action,add1,add,%
      aggregate,alias,Alias,alist,all,anova,any,aov,aperm,append,apply,%
      approx,approxfun,apropos,Arg,args,array,arrows,as,asin,asinh,%
      atan,atan2,atanh,attach,attr,attributes,autoload,autoloader,ave,%
      axis,backsolve,barplot,basename,besselI,besselJ,besselK,besselY,%
      beta,binomial,body,box,boxplot,break,browser,bug,builtins,bxp,by,%
      c,C,call,Call,case,cat,category,cbind,ceiling,character,char,%
      charmatch,check,chol,chol2inv,choose,chull,class,close,cm,codes,%
      coef,coefficients,co,col,colnames,colors,colours,commandArgs,%
      comment,complete,complex,conflicts,Conj,contents,contour,%
      contrasts,contr,control,helmert,contrib,convolve,cooks,coords,%
      distance,coplot,cor,cos,cosh,count,fields,cov,covratio,wt,CRAN,%
      create,crossprod,cummax,cummin,cumprod,cumsum,curve,cut,cycle,D,%
      data,dataentry,date,dbeta,dbinom,dcauchy,dchisq,de,debug,%
      debugger,Defunct,default,delay,delete,deltat,demo,de,density,%
      deparse,dependencies,Deprecated,deriv,description,detach,%
      dev2bitmap,dev,cur,deviance,off,prev,,dexp,df,dfbetas,dffits,%
      dgamma,dgeom,dget,dhyper,diag,diff,digamma,dim,dimnames,dir,%
      dirname,dlnorm,dlogis,dnbinom,dnchisq,dnorm,do,dotplot,double,%
      download,dpois,dput,drop,drop1,dsignrank,dt,dummy,dump,dunif,%
      duplicated,dweibull,dwilcox,dyn,edit,eff,effects,eigen,else,%
      emacs,end,environment,env,erase,eval,equal,evalq,example,exists,%
      exit,exp,expand,expression,External,extract,extractAIC,factor,%
      fail,family,fft,file,filled,find,fitted,fivenum,fix,floor,for,%
      For,formals,format,formatC,formula,Fortran,forwardsolve,frame,%
      frequency,ftable,ftable2table,function,gamma,Gamma,gammaCody,%
      gaussian,gc,gcinfo,gctorture,get,getenv,geterrmessage,getOption,%
      getwd,gl,glm,globalenv,gnome,GNOME,graphics,gray,grep,grey,grid,%
      gsub,hasTsp,hat,heat,help,hist,home,hsv,httpclient,I,identify,if,%
      ifelse,Im,image,\%in\%,index,influence,measures,inherits,install,%
      installed,integer,interaction,interactive,Internal,intersect,%
      inverse,invisible,IQR,is,jitter,kappa,kronecker,labels,lapply,%
      layout,lbeta,lchoose,lcm,legend,length,levels,lgamma,library,%
      licence,license,lines,list,lm,load,local,locator,log,log10,log1p,%
      log2,logical,loglin,lower,lowess,ls,lsfit,lsf,ls,machine,Machine,%
      mad,mahalanobis,make,link,margin,match,Math,matlines,mat,matplot,%
      matpoints,matrix,max,mean,median,memory,menu,merge,methods,min,%
      missing,Mod,mode,model,response,mosaicplot,mtext,mvfft,na,nan,%
      names,omit,nargs,nchar,ncol,NCOL,new,next,NextMethod,nextn,%
      nlevels,nlm,noquote,NotYetImplemented,NotYetUsed,nrow,NROW,null,%
      numeric,\%o\%,objects,offset,old,on,Ops,optim,optimise,optimize,%
      options,or,order,ordered,outer,package,packages,page,pairlist,%
      pairs,palette,panel,par,parent,parse,paste,path,pbeta,pbinom,%
      pcauchy,pchisq,pentagamma,persp,pexp,pf,pgamma,pgeom,phyper,pico,%
      pictex,piechart,Platform,plnorm,plogis,plot,pmatch,pmax,pmin,%
      pnbinom,pnchisq,pnorm,points,poisson,poly,polygon,polyroot,pos,%
      postscript,power,ppoints,ppois,predict,preplot,pretty,Primitive,%
      print,prmatrix,proc,prod,profile,proj,prompt,prop,provide,%
      psignrank,ps,pt,ptukey,punif,pweibull,pwilcox,q,qbeta,qbinom,%
      qcauchy,qchisq,qexp,qf,qgamma,qgeom,qhyper,qlnorm,qlogis,qnbinom,%
      qnchisq,qnorm,qpois,qqline,qqnorm,qqplot,qr,Q,qty,qy,qsignrank,%
      qt,qtukey,quantile,quasi,quit,qunif,quote,qweibull,qwilcox,%
      rainbow,range,rank,rbeta,rbind,rbinom,rcauchy,rchisq,Re,read,csv,%
      csv2,fwf,readline,socket,real,Recall,rect,reformulate,regexpr,%
      relevel,remove,rep,repeat,replace,replications,report,require,%
      resid,residuals,restart,return,rev,rexp,rf,rgamma,rgb,rgeom,R,%
      rhyper,rle,rlnorm,rlogis,rm,rnbinom,RNGkind,rnorm,round,row,%
      rownames,rowsum,rpois,rsignrank,rstandard,rstudent,rt,rug,runif,%
      rweibull,rwilcox,sample,sapply,save,scale,scan,scan,screen,sd,se,%
      search,searchpaths,segments,seq,sequence,setdiff,setequal,set,%
      setwd,show,sign,signif,sin,single,sinh,sink,solve,sort,source,%
      spline,splinefun,split,sqrt,stars,start,stat,stem,step,stop,%
      storage,strstrheight,stripplot,strsplit,structure,strwidth,sub,%
      subset,substitute,substr,substring,sum,summary,sunflowerplot,svd,%
      sweep,switch,symbol,symbols,symnum,sys,status,system,t,table,%
      tabulate,tan,tanh,tapply,tempfile,terms,terrain,tetragamma,text,%
      time,title,topo,trace,traceback,transform,tri,trigamma,trunc,try,%
      ts,tsp,typeof,unclass,undebug,undoc,union,unique,uniroot,unix,%
      unlink,unlist,unname,untrace,update,upper,url,UseMethod,var,%
      variable,vector,Version,vi,warning,warnings,weighted,weights,%
      which,while,window,write,\%x\%,x11,X11,xedit,xemacs,xinch,xor,%
      xpdrows,xy,xyinch,yinch,zapsmall,zip},%
   otherkeywords={!,!=,~,$,*,\%,\&,\%/\%,\%*\%,\%\%,<-,<<-,_,/},%
   alsoother={._$},%
   sensitive,%
   morecomment=[l]\#,%
   morestring=[d]",%
   morestring=[d]'% 2001 Robert Denham
  }%

%%%%%%% 
%% Definitions in yellow boxes
\usepackage{etoolbox}
\setbeamercolor{block title}{use=structure,fg=structure.fg,bg=structure.fg!40!bg}
\setbeamercolor{block body}{parent=normal text,use=block title,bg=block title.bg!20!bg}

\BeforeBeginEnvironment{definition}{%
	\setbeamercolor{block title}{fg=black,bg=yellow!20!white}
	\setbeamercolor{block body}{fg=black, bg=yellow!05!white}
}
\AfterEndEnvironment{definition}{
	\setbeamercolor{block title}{use=structure,fg=structure.fg,bg=structure.fg!20!bg}
	\setbeamercolor{block body}{parent=normal text,use=block title,bg=block title.bg!50!bg, fg=black}
}
\BeforeBeginEnvironment{importanttheorem}{%
	\setbeamercolor{block title}{fg=black,bg=red!20!white}
	\setbeamercolor{block body}{fg=black, bg=red!05!white}
}
\AfterEndEnvironment{importanttheorem}{
	\setbeamercolor{block title}{use=structure,fg=structure.fg,bg=structure.fg!20!bg}
	\setbeamercolor{block body}{parent=normal text,use=block title,bg=block title.bg!50!bg, fg=black}
}
\BeforeBeginEnvironment{importantproperty}{%
	\setbeamercolor{block title}{fg=black,bg=red!50!white}
	\setbeamercolor{block body}{fg=black, bg=red!30!white}
}
\AfterEndEnvironment{importantproperty}{
	\setbeamercolor{block title}{use=structure,fg=structure.fg,bg=structure.fg!20!bg}
	\setbeamercolor{block body}{parent=normal text,use=block title,bg=block title.bg!50!bg, fg=black}
}

% Colour for the outline page
\definecolor{outline_colour}{RGB}{230,165,83}
%% Colours for sections, subsections aand subsubsections
\definecolor{section_colour}{RGB}{27,46,28}
\definecolor{subsection_colour}{RGB}{52,128,56}
\definecolor{subsubsection_colour}{RGB}{150,224,154}
\definecolor{subsub_header_section}{RGB}{196,44,27}
%\definecolor{mygold}{rgb}{1,0.843,0}
% Beginning of a section
% \AtBeginSection[]{
% 	{
% 	  \setbeamercolor{section in toc}{fg=mygold}
% 		\setbeamercolor{background canvas}{bg=section_colour}
% 		\begin{frame}[noframenumbering,plain]
% 			\framesubtitle{\nameofthepart Chapter \insertromanpartnumber \ -- \iteminsert{\insertpart}}
% 			\tableofcontents[
% 				currentsection,
% 				sectionstyle=show/shaded,
% 				subsectionstyle=show/hide/hide,
% 				subsubsectionstyle=hide/hide/hide]
% 		\end{frame}
% 	\addtocounter{page}{-1}
% 	%\addtocounter{framenumber}{-1} 
% 	}
% }


% % Beginning of a section
% \AtBeginSubsection[]{
% 	{
% 	  \setbeamercolor{section in toc}{fg=mygold}
% 		\setbeamercolor{background canvas}{bg=subsection_colour}
% 		\begin{frame}[noframenumbering,plain]
% 				\framesubtitle{\nameofthepart Chapter \insertromanpartnumber \ -- \iteminsert{\insertpart}}
% 				\tableofcontents[
% 					currentsection,
% 					sectionstyle=show/hide,
% 					currentsubsection,
% 					subsectionstyle=show/shaded/hide,
% 					subsubsectionstyle=show/hide/hide]
% 			\end{frame}
% 		\addtocounter{page}{-1}
% 	}
% }

% \newcommand{\newSubSectionSlide}[1]{
% \begin{frame}[noframenumbering,plain]
%   \begin{tikzpicture}[remember picture,overlay]
%     \node[above right,inner sep=0pt,opacity=0.2] at (current page.south west)
%     {
%         \includegraphics[height=\paperheight,width=\paperwidth]{#1}
%     };
%   \end{tikzpicture}
%   \setbeamercolor{section in toc}{fg=subsub_header_section}
%   \setbeamerfont{section in toc}{size=\Large,series=\bfseries}
%   \setbeamertemplate{section in toc shaded}[default][60]
%   \setbeamertemplate{subsection in toc shaded}[default][60]
%   %\setbeamercolor{background canvas}{bg=section_colour}
%   \tableofcontents[
%     currentsection,
%     sectionstyle=show/hide,
%     currentsubsection,
%     subsectionstyle=show/shaded/hide,
%     subsubsectionstyle=show/hide/hide]
% \end{frame}
% \addtocounter{page}{-1}
% }


% % Beginning of a section
% \AtBeginSubsubsection[]{
% 	{
% 	  \setbeamercolor{section in toc}{fg=subsub_header_section}
% 	  \setbeamercolor{subsubsection in toc}{fg=mygold2}
% 	  \setbeamercolor{subsubsection in toc shaded}{fg=mygrey2}
% 		\setbeamercolor{background canvas}{bg=subsubsection_colour}
% 		\begin{frame}[noframenumbering,plain]
% 				\framesubtitle{\nameofthepart Chapter \insertromanpartnumber \ -- \iteminsert{\insertpart}}
% 				\tableofcontents[
% 					currentsection,
% 					sectionstyle=show/hide,
% 					currentsubsection,
% 					subsectionstyle=show/hide/shaded
% 					currentsubsubsection]%,
% 					%subsubsectionstyle=hide/hide/shaded]
% 					%currentsubsubsection]
% 			\end{frame}
% 		\addtocounter{page}{-1}
% 	}
% }


\title[Basic computational analysis]{Basic computational analysis of a mathematical model}
\subtitle{02 -- Using \textit{simulations}}
\author{\texorpdfstring{Julien Arino\newline University of Manitoba\newline\url{julien.arino@umanitoba.ca}}{Julien Arino}}
\date{}
\IfFileExists{upquote.sty}{\usepackage{upquote}}{}
\begin{document}

%%%%%%%%%%%%%%%%%%%%%%%%%%%%%%%%%
%%%%%%%%%%%%%%%%%%%%%%%%%%%%%%%%%
%% TITLE AND OUTLINE
%%%%%%%%%%%%%%%%%%%%%%%%%%%%%%%%%
%%%%%%%%%%%%%%%%%%%%%%%%%%%%%%%%%
\titlepagewithfigure{FIGS/Gemini_Generated_Image_fv8nozfv8nozfv8n.jpeg}

\begin{frame}\frametitle{\emph{Using simulations} ?}
To caricature, suppose we have an IVP $x'=f(x,\mu)$, $x(t_0)=x_0$, where $\mu$ are parameters
\vfill
We can find functional expressions telling us that, say, if $\Psi(\mu)<0$, then the system has a certain behaviour and that this changes when $\Psi(\mu)>0$
\vfill
This type of functional relationship between model parameters and behaviour is what is in the first set of slides for this lecture
\vfill
There are cases also where we need to numerically solve the IVP to obtain, say, the value of an equilibrium $x^\star$, because there is no closed-form formula giving $x^\star$ as a function of $\mu$
\vfill
This type of work \emph{uses simulations} and is what we are interested in here
\end{frame}

\outlinepage{FIGS/Gemini_Generated_Image_fv8np0fv8np0fv8n.jpeg}

%%%%%%%%%%%%%%%%%%%%%%%
%%%%%%%%%%%%%%%%%%%%%%%
%%%%%%%%%%%%%%%%%%%%%%%
%%%%%%%%%%%%%%%%%%%%%%%
\section{Course description}
\newSectionSlide{FIGS/Gemini_Generated_Image_ycrtwoycrtwoycrt.jpeg}

\begin{frame}\frametitle{This is not a \emph{vignette}!}
The vignettes in this repo illustrate how to use \code{R} to consider several problems that a modeller is faced with
\vfill
This is somewhat orthogonal: it takes the information in several of these vignettes and integrates it in the perspective of computational analysis
\vfill
I am including it in this repo, however, because of the non-empty intersection between the two: this is \code{R} and is related to modelling
\end{frame}

\begin{frame}\frametitle{Course objectives}
The objective of the course is to introduce notions used in the computational analysis of a mathematical model
\vfill
This is not an exhaustive course on the subject, but rather an introduction to some basic concepts
\vfill
See this as a minimal toolkit to get you started
\vfill
If you are a graduate student of mine, take this as a hint: this is the type of stuff that I expect to see in your work
\vfill
Note that I am not doing my job properly: I am skipping a very important part of any computational analysis by not doing a proper \emph{return to biology}. This is an essential part of any computational analysis but is outside the scope of these slides
\end{frame}

\begin{frame}\frametitle{Course slides}
These slides are produced using \code{knitr} in \code{Rnw}, i.e., \code{R} within \LaTeX, to illustrate some of the concepts
\vfill
To generate them, you need to have \code{R} and \LaTeX\ installed on your computer and, preferably, \code{RStudio}
\vfill
The code ensures that all the required packages are installed
\end{frame}

\begin{frame}[fragile]\frametitle{Code chunks}
In \code{Rnw} files, code chunks are delimited by \code{<<>>=} and \code{@}

\vfill
Code chunks are highlighted in the RStudio editor, so you should be able to identify them easily
\vfill
I also generate an \code{R} file with all the code (\code{basic-computational-analysis.Rnw}). It is in the CODE directory of the repo. See the last slide for details
\end{frame}


%%%%%%%%%%%%%%%%%%%%%%%
%%%%%%%%%%%%%%%%%%%%%%%
%%%%%%%%%%%%%%%%%%%%%%%
%%%%%%%%%%%%%%%%%%%%%%%
\section{Some toy epidemiological models}
\newSectionSlide{FIGS/Gemini_Generated_Image_fv8np1fv8np1fv8n.jpeg}

\begin{frame}\frametitle{The toy models}
I illustrate the methods using two toy models
\vfill
Both are epidemiological models I have worked on
\vfill
Some of the computational analysis is common to both models, some is specific
\end{frame}

%%%%%%%%%%%%%%%%%%%%%%%
%%%%%%%%%%%%%%%%%%%%%%%
\subsection{The SLIAR epidemic model}
\newSubSectionSlide{FIGS/Gemini_Generated_Image_fv8np3fv8np3fv8n.jpeg}

\begin{frame}\frametitle{The SLIAR epidemic model}
\vskip0pt plus 1filll
Kermack-McKendrick SIR (susceptible-infectious-removed) is a little too simple for many diseases:
\begin{itemize}
\item No incubation period
\item A lot of infectious diseases (in particular respiratory) have mild and less mild forms depending on the patient
\end{itemize}
\vskip0pt plus 1filll
$\implies$ model with SIR but also L(atent) and (A)symptomatic individuals, in which I are now symptomatic individuals
\vskip0pt plus 1filll
\tiny
Arino, Brauer, PvdD, Watmough \& Wu, \href{http://dx.doi.org/10.1098/rsif.2006.0112}{Simple models for containment of a pandemic}, \emph{Journal of the Royal Society Interface} (2006)
\end{frame}


\begin{frame}\frametitle{The SLIAR epidemic model -- Flow diagram}
\centering
\resizebox{\textwidth}{!}{
  \begin{tikzpicture}[%transform canvas={scale=1.3},
      auto,
      cloud/.style={minimum width={width("N-1")+2pt},
      draw, 
      ellipse,
      fill=gray!20}]
    \node [cloud, fill=green!90] (S) {$S$};
    \node [cloud, right=2cm of S, fill=red!30] (L) {$L$};
    \node [cloud, above right=of L, fill=red!90] (I) {$I$};
    \node [cloud, below right=of L, fill=red!60] (A) {$A$};
    \node [cloud, below right=of I, fill=blue!90] (R) {$R$};
    \node [cloud, right=1.5cm of I, draw=none, fill=none] (h1) {};
    %% Infections
    \path [line, very thick] (S) to node [midway, above] (TextNode) {$\beta S(I+\delta A)$} (L);
    \path [line, very thick] (L) to node [midway, above, sloped] (TextNode) {$p\varepsilon L$} (I);
    \path [line, very thick] (L) to node [midway, above, sloped] (TextNode) {$(1-p)\varepsilon L$} (A);
    \path [line, very thick] (I) to node [midway, above, sloped] (TextNode) {$f\gamma I$} (R);
    \path [line, very thick] (A) to node [midway, above, sloped] (TextNode) {$\eta A$} (R);
    \path [line, very thick] (I) to node [midway, above, sloped] (TextNode) {$(1-f)\gamma I$} (h1);
  \end{tikzpicture}
}
\end{frame}

\begin{frame}\frametitle{The SLIAR epidemic model -- Equations}
\begin{subequations}\label{sys:SLIAR}
\begin{align}
S\pprime & = - \beta S (I+\delta A) \\
L\pprime & = \beta S(I+\delta A) - \varepsilon L \\
I\pprime & = p\varepsilon L - \gamma I \\
A\pprime & = p\varepsilon L - \eta A \\
R\pprime & = f\gamma I+\eta A
\end{align}
\end{subequations}
\end{frame}



\begin{frame}\frametitle{The SLIAR epidemic model -- Behaviour}
It's always a good idea to not barge into the computational analysis of a model without an understanding of its behaviour
\vfill
This is an \defword{epidemic} model: all its solutions go to \emph{a} disease-free equilibrium
\vfill
There is a \defword{basic reproduction number} $\R_0$ (next slide) that determines whether the disease will spread or not. If $\R_0<1$, the disease dies out without first going through an outbreakl if $\R_0>1$, the disease goes through an outbreak, then dies out
\vfill
As with many epidemic models, we can also caracterise the so-called \defword{final size} of the epidemic
\end{frame}

\begin{frame}{The SLIAR epidemic model -- Basic reproduction number \& Final size}
We find the basic reproduction number
\begin{equation}\label{eq:R0-SLIAR}
\mathcal{R}_0=\beta
\left(
\frac{p}{\gamma}+\frac{\delta(1-p)}{\eta}
\right)S_0
\end{equation}
\vfill
The final size relation takes the form
\begin{equation}\label{eq:final-size}
S_0(\ln S_0-\ln S_\infty) =
\mathcal{R}_0(S_0-S_\infty)+\frac{\mathcal{R}_0I_0}{\rho}
\end{equation}
with 
\[
\rho=\frac{p}{\gamma}+\frac{\delta(1-p)}{\eta}
\]
\end{frame}

%%%%%%%%%%%%%%%%%%%%%%%
%%%%%%%%%%%%%%%%%%%%%%%
\subsection{The SLIARVS endemic model}
\newSubSectionSlide{FIGS/Gemini_Generated_Image_fv8np2fv8np2fv8n.jpeg}

\begin{frame}\frametitle{The SLIARVS endemic model}
\vskip0pt plus 1filll
The SLIAR model is an epidemic model: all its solutions go to \emph{a} disease-free equilibrium
\vskip0pt plus 1filll
Here we consider a complexification of the SLIAR epidemic model:
\begin{itemize}
\item Add vital dynamics (births and deaths), a.k.a. demography
\item Add a vaccination compartment $V$, with imperfect and waning vaccine
\item Interpret $R$ as \emph{recovered} (and immune) individuals instead of \emph{removed}
\item Add loss of immunity (waning immunity)
\end{itemize}
\vskip0pt plus 1filll
This makes the model \defword{endemic}: it has an endemic equilibrium (EEP) and (roughly) $\R_0$ determines if the system goes to the DFE or the EEP
\vskip0pt plus 1filll
\tiny
Arino \& Milliken, \href{https://doi.org/10.1007/s00285-022-01765-9}{Bistability in deterministic and stochastic SLIAR-type models with imperfect and waning vaccine protection}, \emph{Journal of Mathematical Biology} (2022)
\end{frame}


\begin{frame}\frametitle{The SLIARVS endemic model -- Flow diagram}
\centering
\begin{tikzpicture}[
  scale=1, every node/.style={transform shape},
	auto, %node distance = 2cm, auto,
	box/.style={minimum width={width("N-1")+2pt},
		draw, rectangle,fill=gray!20}]
	%% States
	\node [box] at (0,0) (S) {$S$};
	\node [box] at (2,-3) (V) {$V$};
	\node [box] at (4,0) (L) {$L$};
	\node [box] at (8,2) (I) {$I$};
	\node [box] at (8,-2) (A) {$A$};
	\node [box] at (12,0) (R) {$R$};
	%% Flows
	\path [line, thick, dashed] (-2,0) to node [midway, above] (TextNode) {$(1-v)B$} (S);
	\path [line, thick, dashed] (0,-3) to node [midway, below] (TextNode) {$vB$} (V);
	\path [line, thick] (S) to node [midway, below, sloped] (TextNode) {$\beta S(I+\eta A)$} (L);
	\path [line, thick, bend left] (S) to node [midway, above, sloped] (TextNode) {$eS$} (V);
	\path [line, thick, bend left] (V) to node [sloped, midway, below] (TextNode) {$\omega_v V$} (S);
	\path [line, thick] (V) to node [midway, below, sloped] (TextNode) {$(1-\sigma)\beta V(I+\eta A)$} (L);
	\path [line, thick] (L) to node [sloped, midway, above] (TextNode) {$\pi\varepsilon L$} (I);
	\path [line, thick] (L) to node [sloped, midway, below] (TextNode) {$(1-\pi)\varepsilon L$} (A);
	\path [line, thick] (I) to node [sloped, midway, above] (TextNode) {$\gamma I$} (R);
  \path [line, thick] (A) to node [sloped, midway, below] (TextNode) {$\gamma I$} (R);
	\draw [>=latex,->, thick, rounded corners] (R) -- +(0,2.5) -- ++(-12,2.5) node[sloped, midway, above] {$\omega_r R$} -- (S);
  \path [line, thick, dashed] (S) to node [sloped, midway, above] (TextNode) {$dS$} ++(1.5,1.5);
  \path [line, thick, dashed] (V) to node [sloped, midway, below] (TextNode) {$dV$} ++(0,-1.5);
  \path [line, thick, dashed] (L) to node [sloped, midway, above] (TextNode) {$dL$} ++(0,1.5);
  \path [line, thick] (I) to node [sloped, midway, above] (TextNode) {$(d+\mu)I$} ++(0,-1.75);
  \path [line, thick, dashed] (A) to node [sloped, midway, below] (TextNode) {$dA$} ++(0,-1.5);
  \path [line, thick, dashed] (R) to node [sloped, midway, below] (TextNode) {$dR$} ++(0,-1.5);
\end{tikzpicture}
\end{frame}


\begin{frame}\frametitle{The SLIARVS endemic model -- Equations}
\begin{subequations}\label{sys:SVLIARS}
\begin{align}
S\pprime & = (1-v)B + \omega_vV + \omega_rR - \beta S (I+\eta A) - (e + d) S \\
V\pprime & = vB + e S - (1-\sigma )\beta V (I+\eta A) - (\omega_v + d) V \\
L\pprime & = \beta (S+(1-\sigma)V) (I+\eta A) - (\varepsilon + d) L \\
I\pprime & = \pi\varepsilon L - (\gamma+ \mu + d) I \\
A\pprime & = (1 - \pi)\varepsilon L - (\gamma+ d) A \\
R\pprime & = \gamma (A + I) - (\omega_r + d) R
\end{align}
\end{subequations}
\end{frame}


\begin{frame}\frametitle{The SLIARVS endemic model -- DFE}
In \eqref{sys:SVLIARS} without equation for $V'$ and with $v=e=\omega_v=0$, disease-free equilibrium (DFE) has $\bar S_0=B/d$
\vfill
DFE of full \eqref{sys:SVLIARS} is $E_0 = (S_0,V_0,0,0,0,0)$, where
\begin{equation}\label{eq:DFE-SVLIARS}
S_0 = \frac{(1-v)d+\omega_v}{e+\omega_v+d}\; \frac Bd 
\quad\textrm{and}\quad
V_0 = \frac{vd+e}{e+\omega_v+d}\;\frac Bd
\end{equation}
\end{frame}


\begin{frame}\frametitle{The SLIARVS endemic model -- Reproduction numbers}
With the combination parameter
\begin{equation}\label{eq:lambda}
\lambda = \beta\varepsilon
\frac{(\gamma+\mu+d)\eta(1-\pi) + \pi(\gamma + d)}
{(\gamma + d)(\gamma+\mu+d)}
\end{equation}
we have
\begin{equation}\label{eq:R0-SLIARS}
\mathcal{R}_0 = \frac{\lambda}{\varepsilon+d}\bar S_0
\end{equation}
\begin{equation}\label{eq:Rv-SVLIARS}
\mathcal{R}_v=\frac{\lambda}{\varepsilon+d}(S_0+(1-\sigma)V_0)
\end{equation}
\end{frame}


%%%%%%%%%%%%%%%%%%%%%%%
%%%%%%%%%%%%%%%%%%%%%%%
\subsection{Tackling the models computationally}
\newSubSectionSlide{FIGS/Gemini_Generated_Image_fv8np7fv8np7fv8n.jpeg}

\begin{frame}
The models are systems of ODEs, we could simulate and show the result, but who cares if we just show the behaviour?
\begin{itemize}
\item System \eqref{sys:SLIAR} goes to the DFE every time, after undergoing (or not) an epidemic depending on the value of $\R_0$
\item System \eqref{sys:SVLIARS} goes to the DFE or the EEP, depending on the value of $\R_0$
\end{itemize}
\vfill
Booooriiiing!
\vfill
We can still do things with the solutions, but we'll have to make it worthwhile...
\vfill
To get more insight into the model, we can use the formula for the reproduction numbers \eqref{eq:R0-SLIAR}, \eqref{eq:R0-SLIARS} and \eqref{eq:Rv-SVLIARS}, the final size relation \eqref{eq:final-size} and other quantities to study the model: this will show how these important quantities depend on parameters
\end{frame}


%%%%%%%%%%%%%%%%%%%%%%%
%%%%%%%%%%%%%%%%%%%%%%%
%%%%%%%%%%%%%%%%%%%%%%%
%%%%%%%%%%%%%%%%%%%%%%%
\section{Using simulations of the ODE}
\newSectionSlide{FIGS/Gemini_Generated_Image_1vpyz1vpyz1vpyz1.jpeg}


\begin{frame}
In all we have done so far, $\varepsilon$ and $f$ have not been used
\vfill
Indeed, they do not play a role in the computation of the basic reproduction number $\R_0$ given by \eqref{eq:R0-SLIAR} or the final size relation given by \eqref{eq:final-size}
\vfill
$\varepsilon$ does play a role in determining the ``speed'' of the system, but we have not considered this aspect in our analysis so far
\vfill
$f$ helps determine how many individuals die of the disease and won't be discussed here
\end{frame}


\begin{frame}[fragile]\frametitle{\textsc{Can I have this wrapped up to go?}}
To finish, we use the command \code{purl} to generate an \code{R} file (\code{basic-computational-analysis.R}) in the CODE directory with all the code chunks in this \code{Rnw} file
\vfill
\begin{knitrout}
\definecolor{shadecolor}{rgb}{0.969, 0.969, 0.969}\color{fgcolor}\begin{kframe}
\begin{alltt}
\hlcom{# From https://stackoverflow.com/questions/36868287/purl-within-knit-duplicate-label-error}
\hldef{rmd_chunks_to_r_temp} \hlkwb{<-} \hlkwa{function}\hldef{(}\hlkwc{file}\hldef{)\{}
  \hldef{callr}\hlopt{::}\hlkwd{r}\hldef{(}\hlkwa{function}\hldef{(}\hlkwc{file}\hldef{,} \hlkwc{temp}\hldef{)\{}
    \hldef{out_file} \hlkwb{=} \hlkwd{sprintf}\hldef{(}\hlsng{"../CODE/%s"}\hldef{,} \hlkwd{gsub}\hldef{(}\hlsng{".Rnw"}\hldef{,} \hlsng{".R"}\hldef{, file))}
    \hldef{knitr}\hlopt{::}\hlkwd{purl}\hldef{(file,} \hlkwc{output} \hldef{= out_file,} \hlkwc{documentation} \hldef{=} \hlnum{1}\hldef{)}
  \hldef{\},} \hlkwc{args} \hldef{=} \hlkwd{list}\hldef{(file))}
\hldef{\}}
\hlkwd{rmd_chunks_to_r_temp}\hldef{(}\hlsng{"basic-computational-analysis-2-simulations.Rnw"}\hldef{)}
\end{alltt}
\begin{verbatim}
## [1] "../CODE/basic-computational-analysis-2-simulations.R"
\end{verbatim}
\end{kframe}
\end{knitrout}
\end{frame}

\begin{frame}[fragile]\frametitle{About that \code{R} file}
Source the file \code{basic-computational-analysis-2-simulations.R} (in the \code{CODE} directory) in \code{R} to reproduce all the results in these slides
\vfill
Some small changes are required; for instance, when sourcing (instead of knitting or interactively), \code{ggplot} figures are created but not printed, so in the \code{R} file, you need to print them ``manually''
\vfill
\begin{knitrout}
\definecolor{shadecolor}{rgb}{0.969, 0.969, 0.969}\color{fgcolor}\begin{kframe}
\begin{alltt}
\hldef{pp} \hlkwb{=} \hlkwd{ggplot}\hldef{(...)}
\hlkwd{print}\hldef{(pp)}
\end{alltt}
\end{kframe}
\end{knitrout}
\end{frame}

\end{document}
